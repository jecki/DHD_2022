\begin{document}

\title{Ein LaTeX-Dokument}

\chapter{Kapitel 1}

\section{Ein Abschnitt}

Gallia est omnis divisa in partes tres, quarum unam incolunt Belgae1, aliam
Aquitani2, tertiam qui ipsorum lingua Celtae3, nostra Galli appellantur. Hi
omnes lingua, institutis, legibus inter se differunt. Gallos ab Aquitanis
Garunna4 flumen, a Belgis Matrona5 et Sequana6 dividit. Horum omnium fortissimi
sunt Belgae, propterea quod a cultu atque humanitate provinciae longissime
absunt, minimeque ad eos mercatores saepe commeant atque ea quae ad
effeminandos7 animos pertinent, important, proximique sunt Germanis, qui trans
Rhenum8 incolunt, quibuscum continenter bellum gerunt. Qua de causa Helvetii9
quoque reliquos Gallos virtute praecedunt, quod fere cotidianis proeliis cum
Germanis contendunt, cum aut suis finibus eos prohibent aut ipsi in eorum
finibus bellum gerunt. Eorum una pars, quam Gallos obtinere dictum est, initium
capit a flumine Rhodano10, continetur Garumna flumine, Oceano, finibus Belgarum,
attingit etiam ab Sequanis11 et Helvetiis flumen Rhenum, vergit ad
septentriones12. Belgae ab extremis Galliae finibus oriuntur, pertinent ad
inferiorem partem fluminis Rheni, spectant in septentrionem et orientem solem.
Aquitania a Garunna flumine ad Pyrenaeos13 montes et eam partem Oceani quae est
ad Hispaniam14 pertinet; spectat inter occasum solis et septentriones. 

\subsection{Ein Unterabschnitt}

Apud Helvetios longe nobilissimus fuit et ditissimus1 Orgetorix. Is M. Messala,
[et P.] M. Pisone consulibus regni cupiditate inductus coniurationem nobilitatis
fecit et civitati persuasit ut de finibus suis cum omnibus copiis exirent:
perfacile esse, cum virtute omnibus praestarent, totius Galliae imperio potiri.
Id hoc facilius iis persuasit, quod undique loci natura Helvetii continentur:
una ex parte flumine Rheno latissimo atque altissimo, qui agrum Helvetium a
Germanis dividit; altera ex parte monte Iura2 altissimo, qui est inter Sequanos
et Helvetios; tertia lacu Lemanno3 et flumine Rhodano, qui provinciam nostram ab
Helvetiis dividit. His rebus fiebat ut et minus late vagarentur et minus facile
finitimis bellum inferre possent; qua ex parte homines bellandi cupidi magno
dolore adficiebantur. Pro multitudine autem hominum et pro gloria belli atque
fortitudinis angustos se fines habere arbitrabantur, qui in longitudinem milia
passuum CCXL, in latitudinem CLXXX patebant.

His rebus adducti et auctoritate Orgetorigis1 permoti constituerunt ea quae ad
proficiscendum pertinerent comparare, iumentorum2 et carrorum quam maximum
numerum coemere, sementes quam maximas facere, ut in itinere copia frumenti
suppeteret, cum proximis civitatibus pacem et amicitiam confirmare. Ad eas res
conficiendas biennium3 sibi satis esse duxerunt; in tertium annum profectionem
lege confirmant. Ad eas res conficiendas Orgetorix deligitur. Is sibi legationem
ad civitates suscipit. In eo itinere persuadet Castico, Catamantaloedis filio,
Sequano, cuius pater regnum in Sequanis multos annos obtinuerat et a senatu
populi Romani amicus appellatus erat, ut regnum in civitate sua occuparet, quod
pater ante habuerit; itemque Dumnorigi4 Haeduo, fratri Diviciaci, qui eo tempore
principatum in civitate obtinebat ac maxime plebi acceptus erat, ut idem
conaretur5 persuadet eique filiam suam in matrimonium dat. Perfacile factu esse
illis probat conata perficere, propterea quod ipse suae civitatis imperium
obtenturus esset: non esse dubium quin totius Galliae plurimum Helvetii possent;
se suis copiis suoque exercitu illis regna conciliaturum confirmat. Hac oratione
adducti inter se fidem et ius iurandum dant et regno occupato per tres
potentissimos ac firmissimos populos totius Galliae sese potiri posse sperant. 

\subsection{Ein zweiter Unterabschnitt}

Ea res est Helvetiis per indicium enuntiata. Moribus suis Orgetoricem ex
vinculis causam dicere1 coegerunt; damnatum poenam sequi oportebat, ut igni
cremaretur. Die constituta causae dictionis Orgetorix ad iudicium omnem suam
familiam, ad hominum milia decem, undique coegit, et omnes clientes
obaeratosque2 suos, quorum magnum numerum habebat, eodem conduxit; per eos ne
causam diceret se eripuit. Cum civitas ob eam rem incitata armis ius suum
exsequi conaretur multitudinemque hominum ex agris magistratus cogerent,
Orgetorix mortuus est; neque abest suspicio, ut Helvetii arbitrantur, quin ipse
sibi mortem consciverit. 

\chapter{2. Kapitel}

Zweites Mapitel und Ende

% Hier kommt das Ende

\end{document}
